\pickl{02.05.2024}
\subsection{Eindeutigkeitssatz}
Ziel ist es zu zeigen, dass unter gewissen Bedingungen das Wahrscheinlichkeitsma\ss{} eindeutig festgelegt ist, falls es auf einem Erzeuger gegeben ist.
\\~\\
Wir benutzen das ``Prinzip der guten Mengen''.
\subsection{Satz (Prinzip der guten Mengen)}
Falls eine Eingenschaft f\"ur $\mathcal{E}\subset\mathcal{P}(\Omega)$ gilt und die Menge, auf der die Eigenschaft gilt eine $\sigma$-Algebra ist, so gilt die Eigenschaft auf ganz $\sigma(\mathcal{E})$.
\subsection{Beweis}
Sei $\mathcal{A}$ die Menge, f\"ur die die Eigenschaft gilt. Da $\mathcal{E}\subset\mathcal{A}$, $\mathcal{A}$ ist $\sigma$-Algebra nach Voraussetzung,
\[\sigma(\mathcal{E})=\bigcup_{\mathclap{\mathcal{B}\ \text{ist $\sigma$-Algebra},\ \mathcal{E}\subset\mathcal{B}}}\mathcal{B}.\]
$\mathcal{A}$ ist eines der Kandidaten f\"ur $\mathcal{B}$ $\RA$ $\sigma(\mathcal{E})\subset\mathcal{A}$.
\subsection{Beweisstrategie f\"ur den Eindeutigkeitssatz}
Seien $\PP,\QQ\colon\sigma(\mathcal{E})\to\RR$ Wahrscheinlichkeitsma\ss{}e. Wir m\"ochten zeigen, dass unter gewissen Bedingungen die Menge $\mathcal{G}$ definiert durch
\[A\in\mathcal{G}\LRA\PP(A)=\QQ(A)\]
eine $\sigma$-Algebra ist.
\\~\\
Nach dem Prinzip der guten Mengen ist dann
\[\PP(A)=\QQ(A),\ \forall A\in\sigma(\mathcal{E}).\]
