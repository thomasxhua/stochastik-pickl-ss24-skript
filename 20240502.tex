\pickl{02.05.2024}
\subsection{Eindeutigkeitssatz}
Ziel ist es zu zeigen, dass unter gewissen Bedingungen das Wahrscheinlichkeitsma\ss{} eindeutig festgelegt ist, falls es auf einem Erzeuger gegeben ist.
\\~\\
Wir benutzen das ``Prinzip der guten Mengen''.
\subsubsection{Satz (Prinzip der guten Mengen)}
Falls eine Eingenschaft f\"ur $\mathcal{E}\subset\mathcal{P}(\Omega)$ gilt und die Menge, auf der die Eigenschaft gilt eine $\sigma$-Algebra ist, so gilt die Eigenschaft auf ganz $\sigma(\mathcal{E})$.
\subsubsection{Beweis}
Sei $\mathcal{A}$ die Menge, f\"ur die die Eigenschaft gilt. Da $\mathcal{E}\subset\mathcal{A}$, $\mathcal{A}$ ist $\sigma$-Algebra nach Voraussetzung,
\[\sigma(\mathcal{E})=\bigcup_{\mathclap{\mathcal{B}\ \text{ist $\sigma$-Algebra},\ \mathcal{E}\subset\mathcal{B}}}\mathcal{B}.\]
$\mathcal{A}$ ist eines der Kandidaten f\"ur $\mathcal{B}$ $\RA$ $\sigma(\mathcal{E})\subset\mathcal{A}$.
\subsubsection{Beweisstrategie f\"ur den Eindeutigkeitssatz}
Seien $\PP,\QQ\colon\sigma(\mathcal{E})\to\RR$ Wahrscheinlichkeitsma\ss{}e. Wir m\"ochten zeigen, dass unter gewissen Bedingungen die Menge $\mathcal{G}$ definiert durch
\[A\in\mathcal{G}\LRA\PP(A)=\QQ(A)\]
eine $\sigma$-Algebra ist.
\\~\\
Nach dem Prinzip der guten Mengen ist dann
\[\PP(A)=\QQ(A),\ \forall A\in\sigma(\mathcal{E}).\]
Wir beweisen dies in zwei Schritten:
\num{
\item Die Menge $\mathcal{G}$ ist ein Dynkin-System.
\item Unter gewissen Bedingungen ist jedes Dynkin-System eine $\sigma$-Algebra.
}
\subsubsection{Definition (Dynkin-System)}
Eine Teilmenge $\mathcal{D}$ hei\ss{}t \trt{Dynkin-System} $:\LRA$
\abc{
\item $\Omega\in\mathcal{D}$,
\item $A\in\mathcal{D}\RA A^C\in\mathcal{D}$,
\item $(A_n)_{n\in\NN}\subset\mathcal{D}$, $A_n$ paarweise disjunkt $\RA\bigcup_{n\in\NN}A_n\in\mathcal{D}$.
}
\subsubsection{Satz}
Seien $\PP,\QQ$ Wahrscheinlichkeitsma\ss{}e auf $\mathcal{A}$. $\mathcal{G}:=\{A\subset\mathcal{A}\colon\PP(A)=\QQ(A)\}$ ist ein Dynkin-System.
\subsubsection{Beweis}
\abc{
\item $\PP(\Omega)=1=\QQ(\Omega)\RA\Omega\in\mathcal{G}$.
\item Sei $A\in\mathcal{G}$, d.h. $\PP(A)=\QQ(A)\RA\PP(A^C)=1-\PP(A)=1-\QQ(A)=\QQ(A^C)$, d.h. $A^C\in\mathcal{G}$.
\item Sei $(A_n)_{n\in\NN}\subset\mathcal{G}$ paarweise disjunkt, d.h. $\PP(A_n)=\QQ(A_n),\ \forall n\in\NN$
\[\RA\PP(\bigcup_{n\in\NN}A_n)\textabove{\tbf{K}c)}{=}\Sigma_{n\in\NN}\PP(A_n)\textabove{$\overleftarrow{\text{\tbf{K}c)}}$}{=}\QQ(\bigcup_{n\in\NN}A_n).\]
}
\subsubsection{Definition (Schnittstabilit\"at)}
Eine Menge $\mathcal{E}\subset\mathcal{P}(\Omega)$ hei\ss{}t \trt{schnittstabil} $:\LRA A\cap B\in\mathcal{E},\ \forall A,B\in\mathcal{E}.$
\subsubsection{Satz}
Jedes schnittstabile Dynkin-System ist eine $\sigma$-Algebra.
\subsubsection{Beweis}
Axioma a),b) sind identisch ($\sigma$-Algebra und Dynkin-System). Es bleibt c) zu zeigen. Sei dazu $\mathcal{A}$ ein schnittstabiles Dynkin-System. Sei $(A_n)_{n\in\NN}\subset\mathcal{A}$ beliebig. Z.z. $\bigcup_{n\in\NN}A_n\subset\mathcal{A}$.
\\~\\
Sei $A_1,A_2\in\mathcal{A}\RA A_1\cap A_2\in\mathcal{A}$ (schnittstabil) $\RA(A_1\cap A_2)^C\in\mathcal{A}$. Da $\mathcal{A}$ schnittstabil ist, ist
\[\ubr{(A_1\cap A_2)^C\cap A_1}{$A_1\setminus A_2$}\in\mathcal{A},\ A_1\cup A_2=A_2\cup(A_1\setminus A_2)\in\mathcal{A}\ \text{(disjunkt)}.\]
\[\bigcup_{n\in\NN}A_n=A_1\cup(A_2\setminus A_1)\cup((A_3\setminus A_1)\setminus A_2)\cup\ldots\]
\myqed{}
\subsubsection{Satz}
Ein von $\mathcal{E}\subset\mathcal{P}(\Omega)$ erzeugtes Dynkin-System $\delta(\mathcal{E})$ ist bereits schnittstabil, falls $\mathcal{E}$ schnittstabil ist.
\subsubsection{Beweis}
Sei $\mathcal{E}\subset\mathcal{P}(\Omega)$ schnittstabil, $E\in\mathcal{E}$ beliebig. Sei
\[\mathcal{A}_E:=\{A\in\PP(\Omega)\colon E\cap A\in\delta(\mathcal{E})\}.\]
Wir zeigen nun, dass $\mathcal{A}_E$ ein Dynkin-System ist.
\abc{
\item $\Omega\in\mathcal{A}_E$: $E\cap\Omega=E\in\mathcal{E}\subset\delta(\mathcal{E})$.
\item Sei $A\in\mathcal{A}_E\RA E\cap A\in\delta(\mathcal{E})$:
\[A^C\cap E=(A\cup E^C)^C=((A\cap E)\cup E^C)^C.\]
\item Sei $(A_n)_{n\in\NN}\subset\mathcal{A}_E$ paarweise disjunkt. Z.z. $\bigcup_{n\in\NN}A_n\in\mathcal{A}_E$. $\forall n\in\NN$ ist $A_n\cap E\in\delta(\mathcal{E})$
\[E\cap\bigcup_{n\in\NN}A_n=\bigcup_{n\in\NN}E\cap A_n\ (*).\]
$E\cap A_n$ sind paarweise disjunkt, da die $A_n$ paarweise disjunkt $\RA(*)\in\delta(\mathcal{E})\RA\bigcup_{n\in\NN}A_n\in\mathcal{A}_E$.
}

