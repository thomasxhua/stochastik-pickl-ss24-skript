\pickl{29.04.2024}
\subsection{Borelsche $\sigma$-Algebra}
\subsubsection{Erinnerung}
$\mathcal{B}^\mathbb{R}$ ist die $\sigma$-Algebra erzeugt aus allen offenen Teilmengen von $\mathbb{R}$.
\subsubsection{Bemerkung}
Diese Definition l\"asst sich auf Grundmengen verallgemeinern, in denen es ein ``Konzept'' von offenen Teilmengen gibt (topologische R\"aume, z.B. metrische R\"aume). Wir werden alternative Definitionen von $\mathcal{B}$ betrachten.
\subsubsection{Korollar}
\abc{
\item Sei $\mathcal{E}\subset\mathcal{P}(\Omega),$ $\mathcal{A}\subset\mathcal{P}(\Omega)$ $\sigma$-Algebra. $\mathcal{E}\subset\mathcal{A}\RA\sigma(\mathcal{E})\subset\mathcal{A}$.
\item Seien $\mathcal{E},\mathcal{F}\subset\mathcal{P}(\Omega)$. Falls $\mathcal{E}\subset\sigma(\mathcal{F})$ und $\mathcal{F}\subset\sigma(\mathcal{E})\RA\sigma(\mathcal{E})=\sigma(\mathcal{F})$.
}
\subsubsection{Beweis}
\abc{
\item \[\sigma(\mathcal{E})=\bigcap_{\mathclap{\mathcal{B}\ \sigma\text{-Algebra},\ \mathcal{E}\subset\mathcal{B}}}\mathcal{B}\]
$\leadsto$ $\mathcal{A}$ ist einer der Kandidaten, \"uber die geschnitten wird.
\item
\begin{align*}
\mathcal{E}\subset\sigma(\mathcal{F})&\ \textabove{a)}{\RA}\sigma(\mathcal{E})\subset\sigma(\mathcal{F}),\\
\mathcal{F}\subset\sigma(\mathcal{E})&\ \textabove{a)}{\RA}\sigma(\mathcal{F})\subset\sigma(\mathcal{E}).
\end{align*}
}
\subsubsection{Satz}
Die Borelsche $\sigma$-Algebra $\mathcal{B}^\mathbb{R}$ ist die aus den offenen Intervallen erzeugte $\sigma$-Algebra.
\subsubsection{Beweis}
Z.z.: Jede offene Teilmenge $\mathbb{R}$ liegt in der aus den offenen Intervallen erzeugten $\sigma$-Algebra.
\\~\\
Sei $A\subset\mathbb{R}$ offen. F\"ur jedes $q\in\mathbb{Q}\cap A$ sei
\[r_q=\sup\{\varepsilon\in\mathbb{R}\colon\ ]q-\varepsilon,\ q+\varepsilon[\ \subset A\}.\]
Das Intervall $]q-r_q,\ q+r_q[$ ist $\subset A$ $(*)$. (WA: $]q-r_q,\ q+r_q[\ \not\subset A$ $\RA$ $\exists x\in\ ]q-r_q,\ q+r_q[$ mit $x\notin A$. W\"ahle $\varepsilon=\frac{r_q+|q-x|}{2}$ $\RA$ $x\in\ ]q-r_q,\ q+r_q[$, aber $\varepsilon<r_q$ $\lightning$.)
\\~\\
Betrachte $B=\bigcup_{q\in\mathbb{Q}\cap A}\ ]q-r_q,\ q+r_q[$. Z.z. $A=B$ (da $B$ abz\"ahlbare Vereinigung offener Intervalle).
\bul{
\item $B\subset A$, da all die Teilintervalle $]q-r_q,\ q+r_q[\ \subset A$ $(*)$.
\item $A\subset B$. Sei $y\in A$. Z.z. $y\in B$. Wegen Offenheit von $A$ $\exists\delta_y$, sodass $]y-\delta_y,\ y+\delta_y[\ \subset A$. W\"ahle $q_y\in\mathbb{Q}\cap\ ]y-\frac{\delta_y}{2},y+\frac{\delta_y}{2}[$. $r_{q_y}\geq\frac{\delta_y}{2}$ $\RA$ $y\in\ ]q_y-r_{q_y},\ q_y+r_{q_y}[$.
}
\subsubsection{Satz}
Die Borel-$\sigma$-Algebra ist genau die $\sigma$-Algebra erzeugt aus den Intervallen $]-\infty,\ a]$ mit $a\in\mathbb{R}$.
\subsubsection{Beweis}
$\mathcal{E}:=$ Menge der offenen Intervalle, $\mathcal{F}:=\{]-\infty,\ a],\ a\in\mathbb{R}\}$.
\abc{
\item Z.z. $\mathcal{E}\subset\sigma(\mathcal{F})$.
\\~\\
Sei $a<b\in\mathbb{R}\cup\{\pm\infty\}$. Z.z. $]a,b[\ \in\sigma(\mathcal{F})$.
\bul{
\item 1. Schritt: ``$a=-\infty$''.
\[]-\infty,\ b[\ =\ \ubr{\bigcup_{n\in\mathbb{N}}\ubr{\left]-\infty,\ b-\frac{1}{n}\right]}{$\in\mathcal{F}$}}{$\in\sigma(\mathcal{F})$}.\]
$\forall x\in\ ]-\infty,\ b[$, d.h. $\forall x<b\ \exists n\in\mathbb{N}$, sodass $x<b-\frac{1}{n}\RA x\in\bigcup_{n\in\mathbb{N}}\ \left]-\infty,\ b-\frac{1}{n}\right]$.
$\RA\ ]-\infty,\ b[\ \subset\ \bigcup_{n\in\mathbb{N}}\left]-\infty,\ b-\frac{1}{n}\right]$. $\forall n\in\mathbb{N}$ ist $\left]-\infty,b-\frac{1}{n}\right]\ \subset\ ]-\infty,b[$ $\RA``\supset''$.
\item 2. Schritt:
\[a\in\mathbb{R}\colon\ ]a,b[\ =\ \ubr{]-\infty,b[}{$\in\sigma(\mathcal{F})$ (1. Schritt)}\ \setminus\ \obr{]-\infty,a]}{$\in\mathcal{F}\subset\sigma(\mathcal{F})$}\]
nach Satz ist letzteres in $\sigma(\mathcal{F})$.
}
\item Z.z. $\forall a\in\mathbb{R}$ ist $]-\infty,\ a]\ \in\sigma(\mathcal{E})$.
\[]-\infty,a]\ =\ (]a,\ +\infty[)^C=(\bigcup_{n\in\mathbb{N}}\ ]a,\ n[)^C.\]
}
\subsubsection{Satz}
$\mathcal{B}^\mathbb{R}$ ist die $\sigma$-Algebra erzeugt aus $\{]-\infty,a]\ \text{mit}\ a\in\mathbb{Q}\}$.
\subsubsection{Beweis}
Sei $\mathcal{G}=\{]-\infty,a]\colon a\in\mathbb{Q}\}$. Offensichtlich ist $\mathcal{G}\subset\mathcal{F}$.
\[\mathcal{G}\subset\mathcal{F}\subset\sigma(\mathcal{F}).\]
Z.z. $\mathcal{F}\subset\sigma(\mathcal{G})$. Sei $]-\infty,a]\ a\in\mathbb{R}$.
\[]-\infty,a]\ =\ \ubr{\bigcap_{\mathclap{q\in\mathbb{Q},\ q\geq a}}\ ]-\infty,\ q]}{$\in\sigma(\mathcal{G})$}.\]
$]-\infty,a]\ \subset\ ]-\infty,q]$, $\forall q\geq a$ $\RA$ $]-\infty,a]\subset\bigcup_{q\in\mathbb{Q},\ q\geq a}\ ]-\infty,q]$ ($\RA$ ``$\subset$'').
\\~\\
``$\supset$'': Dazu ``$\subset$'' f\"ur die Komplete. Sei $x\notin]-\infty,\ a]$ $\RA$ $x>a$. W\"ahle $q\in\ ]a,x[\ \cap\mathbb{Q}\RA x\notin\ ]-\infty,\ q]\ \RA x\notin\bigcup_{q\in\mathbb{Q},\ q\geq a}\ ]-\infty,q]$.
