\pickl{10.06.2024}
\subsubsection{Satz}
\abc{
\item Falls $X\geq0$ fast sicher, so ist $\mathbb{E}(x)\geq0$.
\item Falls $X\leq Y$ fast sicher, so ist $\mathbb{E}(x)\leq\mathbb{E}(Y)$.
\item $\mathbb{E}(|X|)=0\LRA X=0$ fast sicher.
\item $\mathbb{E}(|X|)\geq\mathbb{E}(X)$.
}
\subsubsection{Beweis}
\abc{
\item $\mathbb{P}(X\leq0)=0$, $\mathbb{P}(X\leq a)=0$, $\forall a<0$. Die Verteilungsfunktion ist $0$ $\forall a<0$ $\RA$ der negative Anteil von $\mathbb{E}$ ist $0$ $\RA\mathbb{E}(X)\geq 0$.
\item $X\leq Y\RA Y-X\geq 0\RA\mathbb{E}(Y-X)\geq0\RA\mathbb{E}(Y)\geq\mathbb{E}(X)$.
\item
\bul{
\item ``$\LA$'': Sei $X=0$ f.s. $\LRA|X|=0$.
\[
V_X(a)=
\begin{cases}
0&\text{f\"ur }a<0\\
1&\text{f\"ur }a\geq0\\
\end{cases}
\]
$\RA\mathbb{E}(|x|)=0$.
\item ``$\RA$'': Sei $\mathbb{E}(|X|)=0$. WA: $X=0$ f.s. gilt nicht $\RA$ $\exists\varepsilon>0$ mit $\mathbb{P}(|X|>\varepsilon)>0$. Falls $\mathbb{P}(|X|<\frac{1}{n})=0$, w\"are f\"ur alle $n\in\mathbb{N}$, h\"atten wir $\mathbb{P}(\ubr{\bigcup_{n\in\mathbb{N}}|X|<\frac{1}{n}}{$|X|\neq0$})=0$. $\mathbb{P}(|X|>\varepsilon:=p>0\RA V_{|X|}(\varepsilon)=1-p<1\RA\mathbb{E}(|X|)>0$ $\lightning$.
}
\item Es gilt $|X(\omega)|\geq X(\omega),\ \forall\omega$, mit b) folgt Behauptung.
}
\subsection{Varianz}
Wir denken an ein Gl\"ucksspiel. Neben dem Erwartungswert, interessiert uns auch das Risiko.
\\~\\
Das Risiko kann auf unterschiedliche Weise quantifiziert werden. Aufgrund ihrer besonderen Eigenschaften, nehmen wir als Ma\ss{} f\"ur das Risiko die ``Varianz''.
\subsubsection{Definition (Varianz)}
Sei $X$ Zufallsvariable mit $\mathbb{E}(X)=:\mu\in\mathbb{R}$. Dann nennt man
\[
\Var X=\mathbb{E}((X-\mu)^2)
\]
die \trt{Varianz} von $X$.
\subsubsection{Satz}
\abc{
\item $\Var X\geq0$ f\"ur alle Zufallsvariablen $X$.
\item $\Var X=0\LRA X=\mu$ f.s.
\item $\Var(X+a)=\Var X,\ \forall a\in\mathbb{R}$.
\item $\Var(aX)=a^2\cdot\Var X$.
\item $\Var X=\mathbb{E}(X^2)-\mathbb{E}(X)^2$.
}
\subsubsection{Beweis}
\abc{
\item Ist eine Folgerung aus a) des letztes Satzes: $(X-\mu)^2\geq 0$.
\item Ist eine direkte Folgerung asu c) des letzten Satzes, $(X-\mu)^2=0\LRA|X-\mu|=0\RA X-\mu=0$ f.s. $\LRA X=\mu$ f.s.
\item $\Var(X+a)=\mathbb{E}[(X+a-\mathbb{E}(X+a))^2]=\mathbb{E}(X+a-\mathbb{E}(X)-a)^2]=\Var X$.
\item $\Var(aX)=\mathbb{E}[(aX-\mathbb{E}(aX)^2]=\mathbb{E}[(aX-a\mathbb{E}(x))^2]=\mathbb{E}[a^2(X-\mathbb{E}(X))^2]=a^2\mathbb{E}[(X-\mathbb{E}(X))^2]=a^2\Var X$.
\item $\Var X=\mathbb{E}[(X-\mu)^2]=\mathbb{E}(X^2-2\mu X+\mu^2)=\mathbb{E}(X^2)-2\mu\ubr{\mathbb{E}(X)}{$\mu$}+\mu^2=\mathbb{E}(X^2)-\mu^2$.
}
\subsubsection{Definition (Standardabweichung)}
Sei $X$ eine Zufallsvariable mit $\mathbb{E}(X)\in\mathbb{R}$. Dann nennt man
\[\sigma(x):=\sqrt{\Var X}\]
die \trt{Standardabweichung} von $X$.
\subsubsection{Bemerkung}
$\sigma$ misst das ``Risiko'' in den selben Einheiten wie $X$ und $\mu$.
\subsubsection{Beispiele}
\abc{
\item Binomialverteilung: Parameter: $p,n$:
\[
\mathbb{P}(X=k)=\neo{n\\k}p^kq^{n-k}(q=1-p).
\]
\meq{
\mathbb{E}(X)&=\sum_{k=0}^n k\neo{n\\k}p^kq^{n-k}=\sum_{k=0}^n\frac{n!}{k!(n-k)!}p^kq^{n-k}\\
&=\sum_{k=1}^{n}n\cdot\ubr{\frac{\cdot(n-1)!}{(k-1)!(n-k)!}}{$\sum\neo{n-1\\k-1}p^{k-1}q^{n-1}=1$}p\cdot\ubr{p^{k-1}q^{n-k}}{$\leftarrow$}\\
&=n\cdot p.
}
$\Omega=\{0,\ldots,n\}$.
\meq{
\Var X&=\sum_{k=0}^n(k-\mu)^2\neo{n\\k}p^kq^{n-k}\\
\Var X&=\mathbb{E}(\ubr{(X-\mu)^2}{$y$})=\sum_{\omega}y(\omega)\ubr{\mathbb{P}(\{\omega\})}{}\\
&=
}
}

