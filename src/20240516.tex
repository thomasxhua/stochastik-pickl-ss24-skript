\pickl{16.05.2024}
\subsubsection{Schreibweise}
Die durch Zufallsvariablen bestimmten Ereignisse schreibt man oft in kurzer Form:
\bul{
\item $X\leq 5$ beschreibt $\{\omega\in\Omega\colon X(\omega)\leq5\}$.
\item $X\in B$ f\"ur $B\in\mathcal{B}$, $\{\omega\in\Omega\colon X(\omega)\in B\}$.
\item $X\leq Y\cdot2+7$, $\{\omega\in\Omega\colon X(\omega)\in2Y(\omega)+7\}$.
\item $\mathbb{P}(X\leq5)=\mathbb{P}(\{\omega\ldots\})$.
}
\subsection{Unabh\"angigkeit}
\subsubsection{Definition (Unabh\"angigkeit von Zufallsvariablen)}
Zwei reelle Zufallsvariablen $X,Y$ hei\ss{}en \trt{unabh\"angig} $:\LRA$
\[\forall B_1,B_2\in\mathcal{B}_\mathbb{R}\colon X\in B_1\ \text{ist unabh\"angig von}\ Y\in B_2.\]
\subsubsection{Bemerkung}
Es reicht, wenn $B_1,B_2\in\mathcal{E}$ f\"ur einen Erzeuger $\mathcal{E}$ von $\mathcal{B}_\mathbb{R}$ unabh\"angige Ereignisse $X\in B_1$ und $Y\in B_2$ ergeben.
\subsubsection{Beispiel}
Seien $X,Y\colon\Omega\to\{0,1\}$ $\mathcal{A}$-$\mathcal{B}_\mathbb{R}$-messbar.
\begin{center}
\begin{tabular}{c|c|c|c}
$x/y$&$0$&$1$&~\\
\hline
$0$&$p_x\cdot p_y$&$\checkmark$&$p_x$\\
\hline
$1$&$\checkmark$&$\checkmark$&$q_x$\\
\hline
~&$p_y$&$q_y$&~
\end{tabular}
\end{center}
$A$ undba\"angig von $B\RA A$ unabh\"angig von $B^C$.
\subsubsection{Definition (paarweise Unabh\"angigkeit)}
Sei $(A_i)_{i\in\mathcal{I}}$ Ereignisse.
\abc{
\item Die $(A_i)_{i\in\mathcal{I}}$ hei\ss{}en \trt{paarweise unabh\"angig} $:\LRA A_j$ unabh\"angig von $A_k$, $\forall j\neq k\in\mathcal{I}$,
\item die $(A_i)_{i\in\mathcal{I}}$ hei\ss{}en \trt{unabh\"angig} $:\LRA\forall\mathcal{J}\subset\mathcal{I}$ ist $\mathbb{P}(\bigcap_{j\in\mathcal{J}}A_j)=\prod_{j\in\mathcal{J}}\mathbb{P}(A_j)$ ($J$ ist abz\"ahlbar).
}
\subsubsection{Definition (paarweise Unabh\"angigkeit von Zufallsvariablen)}
Seien $(X_i)_{i\in\mathcal{I}}$ Zufallsvariablen.
\abc{
\item Die $(X_i)_{i\in\mathcal{I}}$ hei\ss{}en \trt{paarweise unabh\"angig} $:\LRA X_j,X_k$ unabh\"angig $\forall j\neq k\in\mathcal{I}$,
\item die $(X_i)_{i\in\mathcal{I}}$ hei\ss{}en \trt{unabh\"angig} $:\LRA\forall\mathcal{J}\subset\mathcal{I}$ abz\"ahlbar und $\forall(B_j)_{j\in\mathcal{J}}\subset\mathcal{B}_\mathbb{R}$ gilt $\mathbb{P}(\bigcap_{j\in\mathcal{J}}X_j\in B_j)=\prod_{i\in\mathcal{J}}\mathbb{P}(X_j\in B_j)$.
}
\subsubsection{Bemerkung}
Bei den Anwendungen werden wir meist Situationen betrachten, bei denen a priori klar ist, dass die Zufallsvariablen unabh\"angig sind (\trt{kausale Unabh\"angigkeit}).
\subsubsection{Beispiel}
Seien $X,Y\colon\Omega\to\{0,1\}$ unabh\"angige Zufallsvariablen. Sei
\[Z:=\begin{cases}Z(\omega)=1&\text{falls}\ X(\omega)+Y(\omega)\ \text{ungerade},\\Z(\omega)=0&\text{sonst.}\end{cases}\]
Annahme: $\mathbb{P}(X=1)=\frac{1}{2},\ \mathbb{P}(Y=1)=\frac{1}{2}$.\\
\bul{
\item $\mathbb{P}(Z=1)=\frac{1}{2}$,
\item $\mathbb{P}(X=1\ \text{und}\ Z=1)=\mathbb{P}(X=1\ \text{und}\ Y=0)=\frac{1}{4}$.
}
Nach obiger Vierfeldertafel reicht dies f\"ur die Unabh\"angigkeit von $X$ und $Z$. Die Unabh\"angigkeit von $Y$ und $Z$ ist analog. F\"ur Unabh\"angigkeit m\"usste (u.a.) gelten
\[0=\mathbb{P}(X=1\ \text{und}\ Y=1\ \text{und}\ Z=1)\textabove{?}{=}\mathbb{P}(X=1)\cdot\mathbb{P}(Y=1)\cdot\mathbb{P}(Z=1)=\frac{1}{8}\ \lightning.\]
\subsection{Verteilungsfunktion}
Wir betrachten beliebige reelle Zufallsvariablen, d.h.
\[(\Omega,\mathcal{A},\mathbb{P})\to(\mathbb{R},\mathcal{B}_\mathbb{R},\mathbb{Q}=\mathbb{P}\cdot X^{-1}).\]
Wegen des Eindeutigkeitssatzes ist dieses $\mathbb{Q}$ durch $\mathbb{P}(X\leq a)\ \forall a\in\mathbb{R}$ eindeutig! Letzteres ist eine Funktion in der Variable $a$. Diese nennt man \trt{Verteilungsfunktion}.
\subsubsection{Definition (Verteilungsfunktion)}
$X\colon\Omega\to\mathbb{R}$ Zufallsvariablen.
\[V_X\colon\mathbb{R}\to[0,1]\]
gegeben durch $V_X(a)=\mathbb{P}(X\leq a)$ nennt man die zu $X$ geh\"orige \trt{Verteilungsfunktion}.
\subsubsection{Satz}
Jede Verteilungsfunktion $V_X$ hat folgende Eigenschaften.
\abc{
\item $\lim_{a\to\infty}V_x(a)=1,\ \lim_{a\to-\infty}V_x(a)=0$,
\item $V_X$ ist monoton wachsend,
\item $V_X$ ist rechtsseitig stetig,
\item $V_X$ hat h\"ochstens abz\"ahlbar viele Sprungstellen.
}
\subsubsection{Beispiel}
\abc{
\item $X$ ist Augenzahl eines fairen W\"urfels.
\item $\mathbb{P}\circ X^{-1}$ habe die Dichte
\[
\rho(x)=
\begin{cases}
    1&\text{falls }x\in\ [-\frac{1}{2},\frac{1}{2}],\\
    0&\text{sonst.}
\end{cases}
\]
}
\subsubsection{Definition (Wahrscheinlichkeitsdichte)}
Sei $X$ Zufallsvariable, $B\in\mathcal{B}$. Dann nennt man $\rho\colon\mathbb{R}\to\mathbb{R}$ mit
\[
    \mathbb{P}(X\in B)=\int_{B}\rho(x)dx
\]
eine \trt{Wahrscheinlichkeitsdichte} von $X$.
