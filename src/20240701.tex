\pickl{01.07.2024}
\subsection{Starkes Gesetz der gro\ss{}en Zahlen}
\subsubsection{Satz (Schwaches Gesetz der gro\ss{}en Zahlen, II)}
Sei $(X_n)_{n\to\mathbb{N}}$ eine Folge unabh\"angiger, gleichverteiler, beschr\"ankter Zufallsvariablen. Dann gilt
\[
\mathbb{P}(|\overline{X}_n-\mu|\geq a)\sim e^{-\sqrt{n}}.
\]
\subsubsection{Bemerkung}
Der ``Fehler'' ist also summierbar und damit liefert Borel-Cantelli direkt:
\subsubsection{Satz (Starkes Gesetz der gro\ss{}en Zahlen)}
Unter den obigen Bedigungen ist
\[
\mathbb{P}(\lim_{n\to\infty}\overline{X}_n=\mu)=1.
\]
\subsubsection{Beweis (Schwaches Gesetz der gro\ss{}en Zahlen, II)}
Sei $|X)j|<C,\ \forall j\in\mathbb{N}$. Markov mit $f(x)=e^x$ wird benutzt f\"ur $\overline{X}_n=\frac{1}{n}\sum_{j=1}^nX_j$, o.B.d.A. sei $\mu=0$.
\begin{align*}
\mathbb{P}(|\overline{X}_n|\geq a)\leq\frac{\mathbb{E}(e^{|\overline{X}_n|\sqrt{n}})}{e^{a\sqrt{n}}}
\end{align*}
da $\mathbb{P}(\overline{X_n}\geq a)=\mathbb{P}(\sqrt{n}\overline{X}_n\geq\sqrt{n}a)$. NR (mit $e^{|x|}<e^x+e^{-x}$):
\[
\mathbb{E}(e^{|\overline{X}_n|\sqrt{n}})
<\mathbb{E}(e^{\overline{X}_n\sqrt{n}})
+\mathbb{E}(e^{-\overline{X}_n\sqrt{n}})
\]
und mit
\[
\mathbb{E}(e^{\overline{X}_n\sqrt{n}})
=
\mathbb{E}(\prod_{j=1}^ne^{X_j/\sqrt{n}})
=
\prod_{j=1}^n\mathbb{E}(e^{X_j/\sqrt{n}})
\]
\ldots\weg
\subsubsection{Bemerkung}
Die Anzahl an $(0,1)$-Folgen ist \"uberabz\"ahlbar. Um bereits f\"ur den Fall $X_j\to\{0,1\}$ \"uber eine ganze Folge von $(X_j)_{j\in\mathbb{N}}$ sprechen zu k\"onnen, ben\"otigt man ein \"uberabz\"ahlbares $\Omega$.
\\~\\
Wenn man \"uber das starke Gesetz spricht, muss man also \"uberabz\"ahlbare $\Omega$ behandeln!
\subsection{Zentraler Grenzwertsatz}
Wir betrachten zun\"achst unabh\"angige, identisch verteilt $X_i$, $i\in\{1,\ldots,n\}$ mit $X_i\to\{0,1\}$. Dazu definieren wir:
\begin{itemize}
\item $\overline{X}_n:=\frac{1}{n}\sum_{j=1}^nX_j$,
\item $S_n:=\sum_{j=1}^nX_j$,
\item $N_n:=\frac{1}{\sqrt{n}}\sum_{j=1}^nX_j$.
\end{itemize}
Dann sind
\begin{itemize}
\item $\Var\overline{X}_n=\frac{\sigma^2}{n}$, $\sigma^2=\Var X$,
\item $\Var S_n=\sigma^2\cdot n$,
\item $\Var N_n=\sigma^2$, Standardabweichung$=\sigma$.
\end{itemize}
\subsubsection{Satz (Zentraler Grenzwertsatz, de Moivre-Laplace)}
Seien $X_j,\ j\in\{1,\ldots,n\}$ unabh\"angig, identisch verteilte Zufallsgr\"o\ss{}en, $X_j\to\{0,1\}$ mit $\mathbb{P}(X_j=1)=p$. Dann gilt: $N_n$ konvergiert in Verteilung gegen eine normalverteile Zufallsvariable mit $\mu=p$ und $\sigma^2=pq$ ($q=1-p$).
\subsubsection{Beweisskizze}
Wir zeigen binomial $\to$ normal.
\begin{itemize}
\item Fall $p=q=\frac{1}{2}$:
\[
B(n,\frac{1}{2},k)=\neo{n\\k}\cdot\left(\frac{1}{2}\right)^n\ (S_N).
\]
Betrachte $f(x):=\neo{n\\\frac{n}{2}+x}$ f\"ur $x\in[-\sqrt{n},\sqrt{n}]$ bzw. $g(x)=\ln f(x)=\ldots$ \weg
\end{itemize}
