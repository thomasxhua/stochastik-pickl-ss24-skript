\pickl{13.05.2024}
\subsubsection{Satz}
$X$ Zufallsvariable $\Rightarrow\mathbb{Q}:=\mathbb{P}\cdot X^{-1}$ ist Wahrscheinlichkeitsma\ss{}.
\subsubsection{Erinnerung}
$X^{-1}:=\{\omega\in\Omega\colon X(\omega)\in A\},\ X\colon\Omega\to\vartheta$.
\subsubsection{Proposition}
Sei $X\colon\Omega\to\vartheta$ eine Abbildung. $A,B\subset\vartheta$. Dann gilt:
\abc{
\item $X^{-1}(\vartheta)=\Omega$,
\item $X^{-1}(A^C)=[X^{-1}(A)]^C$,
\item $X^{-1}(A)\cup X^{-1}(B)=X^{-1}(A\cup B)$,
\item $X^{-1}(A)\cap X^{-1}(B)=X^{-1}(A\cap B)$,
\item $X^{-1}(A)\cap X^{-1}(B)=\emptyset$, falls $A\cap B=\emptyset$.
}
\subsubsection{Beweis}
\abc{
\item $\forall\omega\in\Omega$ gilt $X(\omega)\in\vartheta$.
\item $\omega\in X^{-1}(A^C)\Leftrightarrow X(\omega)\in A^C\Leftrightarrow X(\omega)\notin A\Leftrightarrow\omega\notin X^{-1}(A)\Leftrightarrow w\in[X^{-1}(A)]^C$.
\item $\omega\in X^{-1}(A)\cup X^{-1}(B)\Leftrightarrow\omega\in X^{-1}(A)$ oder $\omega\in X^{-1}(B)\Leftrightarrow X(\omega)\in A$ oder $X(\omega)\in B\Leftrightarrow \omega\in X^{-1}(A\cup B)$. Geht ebenso f\"ur beliebige, insbesondere abz\"ahlbare Vereinigungen.
\item folgt aus mehrfachem Anwenden von b), und c).
\item folgt aus d), da $X^{-1}(\emptyset)=\emptyset$.
}
\subsubsection{Beweis des Satzes}
\abc{
\item $\mathbb{Q}(\vartheta):=\mathbb{P}(X^{-1}=\mathbb{P}(\Omega)=1$.
\item Positivit\"at folgt aus b) f\"ur $\mathbb{P}$.
\item $\mathbb{Q}(\bigcup_{j=1}^\infty A_j)$ f\"ur $A_j$ paarweise disjunkt.
\[=\mathbb{P}(X^{-1}(\bigcup_{j=1}^\infty A_j))=\mathbb{P}(\bigcup_{j=1}^\infty\ubr{X^{-1}(A_j)}{paarweise disjunkt e)})=\sum_{j=1}^\infty\mathbb{P}(X^{-1}(A_j))=\sum_{j=1}^\infty\mathbb{Q}(A_j).\]
}
\subsubsection{Satz}
Sei $X\colon\Omega\to\vartheta$ messbar ($\mathcal{A}\to\mathcal{B}$), dann sind folgenden Mengen $\sigma$-Algebren:
\abc{
\item $X^{-1}(\mathcal{B}):=\{X^{-1}(B)\colon B\in\mathcal{B}\}$,
\item $X_*(\mathcal{A}):=\{B\in\mathcal{B}\colon\exists A\in\mathcal{A}\colon X^{-1}(B)=A\}.$
}
a) bzgl. $\Omega$, b) bzgl. $\vartheta$.
\subsubsection{Beweis}
\abc{
\item
\abc{
\item $\Omega\in X^{-1}(B)$, da $\Omega=X^{-1}(\vartheta)$.
\item Sei $A\in X^{-1}(B)\Rightarrow\exists B$ mit $X^{-1}(B)=A\Rightarrow X^{-1}(B^C)=A^C\in X^{-1}(\mathcal{B})$.
\item Seien $(A_n)_{n\in\mathbb{N}}\subset X^{-1}(\mathcal{B})$
\begin{align*}
&\ \Rightarrow\forall n\in\mathbb{N}\ \exists B_n\in\mathcal{B}\ \text{mit}\ A_n=X^{-1}(B_n)\\
&\ \Rightarrow\bigcup_{n\in\mathbb{N}}B_n\in B\ \text{und}\ X^{-1}(\bigcup_{n\in\mathbb{N}}B_n)=\bigcup_{n\in\mathbb{N}}X^{-1}(B_n)\\
&\ \Rightarrow\bigcup_{n\in\mathbb{N}}A_n=\bigcup_{n\in\mathbb{N}}X^{-1}(B_n)=X^{-1}(\ubr{\bigcup_{n\in\mathbb{N}}B_n}{$\in\mathcal{B}$})\\
&\ \Rightarrow\bigcup_{n\in\mathbb{N}}A_n\in X^{-1}(\mathcal{B}).
\end{align*}
}
\item \"ahnlich.
}
\subsubsection{Bemerkung}
Die Messbarkeit ging in den Beweis gar nicht ein. F\"ur jede Abbildung $X\colon\Omega\to\vartheta$ gilt, dass f\"ur eine $\sigma$-Algebra $\mathcal{B}$ bzgl. $X^{-1}(\mathcal{B}$ eine $\sigma$-Algebra bzgl. $\Omega$ ist.
\subsubsection{Korollar}
Sei $X\colon\Omega\to\vartheta$ eine Abbildung, $\mathcal{B}$ eine von $\mathcal{E}$ erzeugte $\sigma$-Algebra bzgl. $\vartheta$. Dann ist $X$ $\mathcal{A}$-$\mathcal{B}$-messbar bereits., wenn $X^{-1}(E)\in\mathcal{A}$, $\forall E\in\mathcal{E}$ $(*)$.
\subsubsection{Beweis}
Wir betrachten $X_*(\mathcal{A})$. Falls $(*)$ gilt, ist $\mathcal{E}\subset X_*(\mathcal{A})$. $X_*(\mathcal{A})$ ist $\sigma$-Algebra $\Rightarrow B\subset X_*(\mathcal{A})\Rightarrow X$ ist $\mathcal{A}$-$\mathcal{B}$-messbar.
\subsubsection{Beispiel}
Jede stetige Funktion $f\colon\mathbb{R}\to\mathbb{R}$ ist $B_\mathbb{R}$-$B_\mathbb{R}$-messbar. Grund: Da $f$ stetig, sind die Urbilder offener Mengen offen. Die offenen Mengen sind ein m\"oglicher Erzeuger von $B_\mathbb{R}$, nach Korollar muss man nur diese untersuchen.
\subsubsection{Satz}
Seien $\Omega,\vartheta,\Lambda$ Mengen, $\mathcal{A},\mathcal{B},\mathcal{C}$ $\sigma$-Algebren bzgl. $\Omega,\vartheta,\Lambda$. Sei $X\colon\Omega\to\vartheta$ $mathcal{A}$-$\mathcal{B}$-messbar, $Y\colon\vartheta\to\Lambda$ $\mathcal{B}$-$\mathcal{C}$ messbar. Dann ist $Z:=Y\cdot X\colon\Omega\to\Lambda$ $\mathcal{A}$-$\mathcal{C}$-messbar.
\subsubsection{Beweis}
Z.z.: $Z^{-1}(A)\in\mathcal{A}\ \forall A\in\mathcal{C}$.
\[Z^{-1}(A)=\ubr{X^{-1}(\obr{Y^{-1}(A)}{$\in\mathcal{B}$, da $Y$ messbar})}{$\in\mathcal{A}$}.\]
\subsubsection{Satz}
Seien $X,Y\colon\Omega\to\mathbb{R}$ $\mathcal{A}$-$\mathcal{B}_\mathbb{R}$ messbar. Dann ist $X+Y$ ebenfalls $\mathcal{A}$-$\mathcal{B}_\mathbb{R}$ messbar.
\subsubsection{Beweis}
Wegen des Korollars gilt es zu zeigen:
\[(X+Y)^{-1}(]-\infty,a[)\in\mathcal{A},\ \forall a\in\mathbb{R}.\]
Betrachte dazu:
\[\ubr{\bigcup_{b\in\mathbb{Q}}\ubr{X^{-1}(\obr{]-\infty,b[}{$\in\mathcal{B}_\mathbb{R}$})}{$\in\mathcal{A}$}\ \cap\ \ubr{Y^{-1}(\obr{]-\infty,a-b[}{$\in\mathcal{B}_\mathbb{R}$})}{$\in\mathcal{A}$}}{$\in\mathcal{A}$}.\]
Wir zeigen nun, dass
\[(X+Y)^{-1}(]-\infty,a[)=\bigcup_{b\in\mathbb{Q}}X^{-1}(]-\infty,b[)\ \cap\ Y^{-1}(]-\infty,a-b).\]
\bul{
\item ``$\subset$'': Sei $\omega\in(X+Y)^{-1}(]-\infty,a[)\Leftrightarrow(X+Y)(\omega)<a$. Sei $b\in\mathbb{Q}$ mit $b\in]X(\omega),X(\omega)+\mathcal{\varepsilon}{2}[$. $\Rightarrow X(\omega)<b$ und $Y(\omega)+b<X(\omega)+Y(\omega)+\frac{\varepsilon}{2}<a\Rightarrow Y(\omega)<a-b$.
\item ``$\supset$'': Sei $\omega\in\ldots\Rightarrow\exists b$, sodass $X(\omega)<b$ und $Y(\omega)<a-b\Rightarrow(X+Y)(\omega)<a$.
}
