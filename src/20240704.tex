\pickl{04.07.2024}
\subsubsection{Satz (Zentraler Grenzwertsatz)}
Sei $(X_n)_{n\in\mathbb{N}}$ eine Folge unabh\"angiger und identsich verteilter (i.i.d.) Zufallsvariablen mit endlichem Erwartungswert mit endlichenm Erwartungswert und endlicher Varianz. Dann konvergiert
\[
N_n=\frac{1}{\sqrt{n}}\sum_{j=1}^n(X_j-\mu)
\]
in Verteilung gegen eine Normalverteilung mit $\mu=0$ und $\sigma^2=\Var X_j$.
\newpage
\section{Statistik}
\subsection{Maximum Likelihood}
Gegeben sei ein Zufallsexperiment. Anhand vom Ausgang des Experiments m\"ochten wir die Parameter der entsprechenden Verteilungsfunktion sch\"atzen. Wir nehmen jene Sch\"atzwerte f\"ur die Parameter, unter denen das Resultet maximale Wahrscheinlichkeitsdichte hat.
\subsubsection{Beispiel}
Eine Versicherung m\"ochte die Verteilungsfunktion f\"ur E-Scooter-Unf\"alle pro Jahr in T\"ubingen absch\"atzen. Zun\"achst ist recht klar, dass es sich um eine Poisson-Verteilung handelt. Die Versicherung erh\"alt folgende Daten:
\begin{center}
\begin{tabular}{c|c|c|c}
Jahr&$1$&$2$&$3$\\
\hline
Unf\"alle&$2$&$0$&$5$\\
\end{tabular}
\end{center}
$\mathbb{P}(X=k)=e^{-\lambda}\cdot\frac{\lambda^k}{k!}$. Dann ist
\[
\mathbb{P}(X_1=2;\ X_2=0;\ X_3=5)=
e^{-\lambda}\cdot\frac{\lambda^2}{2!}
\cdot
e^{-\lambda}\cdot\frac{\lambda^0}{0!}
\cdot
e^{-\lambda}\cdot\frac{\lambda^5}{5!}
=e^{-3\lambda}\cdot\frac{\lambda^7}{2!5!}=:f(x).
\]
Wir suchen das Maximum dieser Funnktion bzw. $\lambda$, sodass dies maximal wird.
\[
f^\prime(\lambda)=\ldots=e^{-3\lambda}\lambda^6(-3\lambda+7)\Rightarrow\lambda=\frac{7}{3}.
\]
