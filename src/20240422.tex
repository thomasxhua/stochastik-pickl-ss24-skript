\pickl{22.04.2024}
\subsubsection{Lemma}
Es gibt kein translationsinvariantes Wahrscheinlichkeitsma\ss{} auf $\PP([0,2\pi[)$.
\subsubsection{Beweis}
WA: Es gibt ein solches $\PP$.
\\~\\
Wir zerlegen nun $[0,2\pi[$ in \"uberabz\"ahlbar viele abz\"ahlbare Teilmengen:
\meq{a~b:\LRA a-b\in\QQ.}
Dies definiert \"Aquivalenzklassen, diese dienen zur Zerlegung von $[0,2\pi[$:
\bul{
\item $\{0,\frac{1}{2},\frac{1}{3},1,\ldots\}$,
\item $\{\sqrt{2},\sqrt{2}+\frac{1}{2},\ldots\}$,
\item $\{\pi,\pi+\frac{1}{2},\ldots\}$,
\item \ldots
}
Wir w\"ahlen aus jeder \"Aquivalenzklasse einen Representanten $\leftarrow\{0,\sqrt{2},\pi,\ldots\}$. $R$ ist nat\"urlich \"uberabz\"ahlbar. F\"ur jedes $q\in[0,2\pi[\cap\QQ$ definieren wir
\meq{R_q:=\{r+q,\ r\in R\}=R+q.}
Nach Annahme der Translationsinvarianz ist
\meq{\PP(R_q)=\PP(R),\ \forall q\in[0,2\pi[\cap\QQ.}
Es gilt:
\meq{{}[0,2\pi[\ \subseteq\ \bigcup_{\mathclap{q\in]-2\pi,2\pi[\cap\QQ}}\ R_q\subset\ ]-2\pi,4\pi[.}
$\RA$ da $\PP([0,2\pi[)=1$, gilt, dass
\meq{
    &\PP(\bigcup_{\mathclap{q\in]-2\pi,2\pi[\cap\QQ}}R_q)\geq1.\\
    \RA\ &\sum_{\mathclap{j\in]-2\pi,2\pi[}}\PP(R)\geq1\\
    \RA\ &\PP(R)\neq0,
}
aber falls der Inhalt $\PP(R)>0$, folgt, dass das Intervall $]-2\pi,4\pi[$ unendlichen Inhalt hat. Dieses \"uberdeckt jedoch $[0,2\pi[$ dreimal. $\lightning$
\subsection{Ereignisraum}
Wir schr\"anken den Definitionsbereich von $\PP$ ein, um solch problematische Mengen zu umgehen. 
\subsubsection{$\sigma$-Algebra}
Sei $\Omega$ eine Menge. Eine $\sigma$-Algebra bzgl. $\Omega$ ist eine Teilmenge von $\mathcal{P}(\Omega)$ mit:
\abc{
\item $\Omega\in\mathcal{A}$,
\item $A\in\mathcal{A}\RA A^C\in\mathcal{A}$,
\item Sei $(A_n)_{n\in\NN}\subset\mathcal{A}\RA\bigcup_{n=1}^\infty A_n\in\mathcal{A}$.
}
\subsubsection{Beispiele}
\abc{
\item $\mathcal{P}(\Omega)$, sowie $\{\emptyset,\Omega\}$ ist jeweils $\sigma$-Algebra.
\item
    \bul{
    \item $\Omega=\{1,2,3,4\}$.
    \item $\mathcal{A}=\{\emptyset,\Omega,\{1,2\},\{3\},\{4\},\{3,4\},\{1,2,4\},\{1,2,3\}\}$.
    }
}
\subsubsection{Satz}
Sei $\mathcal{A}$ eine $\sigma$-Algebra bzgl. $\Omega$. Dann gilt:
\abc{
\item $\emptyset\in\mathcal{A}$,
\item abz\"ahlbare Schritte von Ereignissen sind in $\mathcal{A}$,
\item $A,B\in\mathcal{A}\RA A\setminus B\in\mathcal{A}$.
}
\subsubsection{Beweis}
\abc{
\item $\emptyset=\Omega^C\in\mathcal{A}$,
\item $\bigcap_{n=1}^\infty A_n=\left[\bigcup_{n=1}^\infty A^C_n\right]^C\in\mathcal{A}$,
\item $A\setminus B=A\cap B^C$.
}
\subsubsection{Satz}
Sei $\Omega$ eine Menge. Der Schnitt beliebiger $\sigma$-Algebren ergibt wieder eine $\sigma$-Algebra.
\subsubsection{Beweis}
Seien $\mathcal{A}_i$ f\"ur $i\in\mathcal{I}$ $\sigma$-Algebren (bzgl. $\Omega$). Z.z. $\mathcal{A}:=\bigcap_{i\in\mathcal{I}}\mathcal{A}_i$ ist $\sigma$-Algebra.
\abc{
\item $\Omega\in\mathcal{A}_i,\ \forall i\in\mathcal{I}\RA\Omega\in\mathcal{A}.$
\item Sei $E\in\mathcal{A}\RA E\in\mathcal{A}_i,\ \forall i\in\mathcal{I}\RA E^C\in\mathcal{A}_i,\ \forall i\in\mathcal{I}$.
\item Seien $(E_n)_{n\in\NN}\subset\mathcal{A}$ ($E_n\in\mathcal{A}\ \forall n\in\NN$):
\meq{
    \RA\ &E_n\in A_i,\ \forall n\in\NN, \forall i\in\mathcal{I}\\
    \RA\ &\bigcup_{n=1}^\infty E_n\in\mathcal{A}_i,\ \forall i\in\mathcal{I}\ \text{(da $\mathcal{A}_i$ $\sigma$-Algebra)}\\
    \RA\ &\bigcup_{n=1}^\infty E_n\in\mathcal{A}.
}
}
\subsubsection{Bemerkung}
Vereinigungen von $\sigma$-Algebren ergeben \trt{nicht} notwendigerweise eine $\sigma$-Algebra.
\subsubsection{Beispiel}
$\Omega=\{1,2,3\}$.
\bul{
\item $\mathcal{A}_1=\{\Omega,\emptyset,\{1,2\},\{3\}\}$,
\item $\mathcal{A}_2=\{\Omega,\emptyset,\{1,3\},\{2\}\}$,
\item $\mathcal{A}_1\cup\mathcal{A}_2=\{\Omega,\emptyset,\{1,2\},\{1,3\},\{2\},\{3\}\}\not\ni\{2\}\cup\{3\}.$
}
\subsubsection{Definition (Erzeugte $\sigma$-Algebra)}
Sei $\Omega$ eine Menge, $\mathcal{E}\subset\mathcal{P}(\Omega)$. Die $\sigma$-Algebra $\sigma(\mathcal{E})$ definiert durch
\meq{\sigma(\mathcal{E}):=\bigcap_{\mathclap{\mathcal{A}\text{ ist $\sigma$-Alg},\ \mathcal{E}\subset\mathcal{A}}}\mathcal{A}}
nennen wir die \trt{von $\mathcal{E}$ erzeugte $\sigma$-Algebra}.
\subsubsection{Bemerkung}
$\sigma(\mathcal{E})$ ist f\"ur nichtleere $\Omega$ immer wohldefiniert und wegen Satz $\sigma$-Algebra.
\subsubsection{Korollar}
$\sigma(\mathcal{E})$ ist die kleinste $\sigma$-Algebra, die $\mathcal{E}$ enth\"alt, d.h.
\abc{
\item $\mathcal{E}\in\sigma(\mathcal{E})$,
\item $\forall\widetilde{\mathcal{A}}$ $\sigma$-Algebra it $\mathcal{E}\subset\widetilde{\mathcal{A}}$, gilt $\sigma(\mathcal{E})\subset\widetilde{\mathcal{A}}$.
}
\subsubsection{Beweis}
\abc{
\item $\mathcal{E}$ ist in allen $\sigma$-Algebren enthalten, \"uber die der Schnitt gebildet wird.
\item $\widetilde{\mathcal{A}}$ ist ein Kandidat f\"ur $\mathcal{A}$ in der Definition. Es wird also auch \"uber $\widetilde{\mathcal{A}}$ der Schnitt gebildet $\RA\widetilde{A}\supset\sigma(\mathcal{E})$.
}
\subsubsection{Beispiel}
$\Omega=\{1,2,3\},\mathcal{E}=\{\{1,2\}\}$. $\mathcal{P}(\Omega)$ ist $\sigma$-Algebra, enth\"alt $\mathcal{E}$ $\mathcal{A}_1$ vom Beispiel oben ebenso. Weitere Kandidaten:
\meq{\mathcal{B}=\{\Omega,\emptyset,\{1,2\},\{3\},\{1\},\{2\},\ldots\}.}

