\pickl{18.07.2024}
\subsubsection{$\chi^2$-Verteilung}
\tul{Annahme:} $X_1,\ldots,X_n$ unabh\"angig und standardnormalverteilt.\\
\tul{Frage:} Wie sieht die Verteilungsfunktion von $Z:=\sum_{j=1}^nX_j^2$ aus?
\subsubsection{Definition ($\chi^2$-Verteilung)}
Die Verteilungsfunktion von $Z$ nennt man \trt{chi-Quadrat-Verteilung} mit $n$ Freiheitsgraden (man schreibt $\chi^2(n)$).
\subsubsection{Berechnung der Wahrscheinlichkeitsdichte von $Z$}
\[
f_{X_1,\ldots,X_n}(x_1\ldots x_n)=\prod_{j=1}^n\frac{e^{-\frac{1}{2}x_j^2}}{\sqrt{2\pi}}=(2\pi)^{-\frac{n}{2}}e^{-\frac{1}{2}\sum_{j=1}^nx_j^2},
\]
\ldots
\begin{align*}
\rho_n(z)
&=\lim_{h\to0}\frac{1}{h}\mathbb{P}(\obr{z<Z\leq z+h}{$\mathbb{P}(Z\leq z+h)-\mathbb{P}(Z\leq z)=V(z+h)-V(z)$})\\
&=\lim_{h\to0}\frac{1}{h}\int_M(2\pi)^{-\frac{n}{2}}e^{-\frac{1}{2}\sum_{j=1}^nx_j^2}\mathrm{d}x_1\ldots x_n\\
&=(2\pi)^{-\frac{n}{2}}e^{-\frac{1}{2}z}\ubr{\lim_{h\to0}\frac{1}{n}\int_M\mathrm{d}x_1\ldots x_n}{$\frac{\mathrm{d}}{\mathrm{dz}} V_n(\sqrt{n})=\ldots\frac{z^{\frac{n}{2}-1}\cdot\pi^{\frac{n}{2}}}{\Gamma(\frac{n}{2})}$}\\
&=
\frac{1}{2^{\frac{n}{2}}\Gamma(\frac{n}{2})}z^{\frac{n}{2}-1}e^{-\frac{z}{2}},\ z>0,\ \text{sonst ist }\rho=0.
\end{align*}
mit $M=\{z\leq\ubr{\sum_{j=1}^nx_j^2}{$\sim z$}\leq z+h\}$.
\subsubsection{$\chi^2$-Verteilungtest}
\tul{Motivation:} W\"urfel
\begin{center}
\begin{tabular}{c|c|c|c|c|c|c}
Wert&1&2&3&4&5&6\\
\hline
H\"aufigkeit&$n_1$&$n_2$&$n_3$&$n_4$&$n_5$&$n_6$
\end{tabular}
\end{center}
Wir betrachten dazu die Gr\"o\ss{}e
\[
\sum_{j=1}^k\frac{(nP_j-n_j)^2}{nP_j}
\]
mit $n=\sum_{j=1}^kn_j$.
\subsubsection{Bemerkung}
Der obige Ausdruck $A$ ist $\chi^2$-verteilt mit $k-1$ Freiheitsgraden (vorausgesetzt $P_j$ korrekt). Erinnerung: $Z=\sum_{j=1}^{k-1}X_j^2$, $X_j$ standardnormalverteilt und unabh\"angig.
\\~\\
Fall $k=2$. Betrachte $Z=\frac{(nP_1-n_1)}{\sqrt{np_1q_1}}$. Falls meine M\"unze in der Tat Wahrscheinlichkeit ``p'' f\"ur Kopf hat, ist das standardnormalverteilt. Also $Z^2$ ist $X^2(1)$-verteilt.
\[
Z^2=\frac{(np_1-n_1)^2}{np_1q_1}=\frac{(np-n_1)^2(p+q)}{npq}=\frac{(np-n_1)^2}{nq}+\frac{(np-n_1)^2}{np}=\ldots=\sum_{j=1}^2\frac{(np_j-n_j)^2}{np_j}.
\]
