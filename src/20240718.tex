\pickl{18.07.2024}
\subsubsection{$\chi^2$-Verteilung}
\tul{Annahme:} $X_1,\ldots,X_n$ unabh\"angig und standardnormalverteilt.\\
\tul{Frage:} Wie sieht die Verteilungsfunktion von $Z:=\sum_{j=1}^nX_j^2$ aus?
\subsubsection{Definition ($\chi^2$-Verteilung)}
Die Verteilungsfunktion von $Z$ nennt man \trt{chi-Quadrat-Verteilung} mit $n$ Freiheitsgraden (man schreibt $\chi^2(n)$).
\subsubsection{Berechnung der Wahrscheinlichkeitsdichte von $Z$}
\[
f_{X_1,\ldots,X_n}(x_1\ldots x_n)=\prod_{j=1}^n\frac{e^{-\frac{1}{2}x_j^2}}{\sqrt{2\pi}}=\ldots
\]
\ldots
\[
\rho_n(z)=\frac{1}{2^{\frac{n}{2}}\Gamma(\frac{n}{2})}z^{\frac{n}{2}-1}e^{-\frac{z}{2}},\ z>0,\ \text{sonst ist }\rho=0.
\]
\subsubsection{$\chi^2$-Verteilungtest}
\tul{Motivation:} W\"urfel
\begin{center}
\begin{tabular}{c|c|c|c|c|c|c}
Wert&1&2&3&4&5&6\\
\hline
H\"aufigkeit&$n_1$&$n_2$&$n_3$&$n_4$&$n_5$&$n_6$
\end{tabular}
\end{center}
Wir betrachten dazu die Gr\"o\ss{}e
\[
\sum_{j=1}^k\frac{(nP_j-n_j)^2}{nP_j}
\]
mit $n=\sum_{j=1}^kn_j$.
\subsubsection{Bemerkung}
Der obige Ausdruck $A$ ist $\Xi^2$-verteilt mit $k-1$ Freiheitsgraden (vorausgesetzt $P_j$ korrekt). Erinnerung: $Z=\sum_{j=1}^{k-1}X_j^2$.
\\~\\
Fall $k=2$. Betrachte $Z=\frac{(nP_1-n_1)^2}{np_1q_1}$. Falls meine M\"unze in der Tat Wahrscheinlichkeit ``p'' f\"ur Kopf hat, ist das standard-normalverteilt. ALso $Z^2$ ist $X^2(1)$-verteilt.
\[
Z^2=
\]
% mid https://timms-ms09.uni-tuebingen.de/UT_2024/07/18/UT_20240718_002_sose24stochastik_0001.854x480_cb950.mp4
% teil2 to go
