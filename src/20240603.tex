\pickl{03.06.2024}
\subsubsection{Definition (Normalverteilung)}
Eine Zufallsgr\"o\ss{}e $X$ mit der Wahrscheinlichkeitsdichte
\[\rho(x):=\frac{1}{C}\cdot e^{-\frac{(x-\mu)^2}{2\sigma^{2}}},\]
f\"ur die Parameter $\mu\in\mathbb{R}$, $\sigma\in\mathbb{R}^+$ und $C$, sodass $\int_{-\infty}^{\infty}\rho(x)\,dx=1$ nennt man \trt{normalverteilt}.
\\~\\
Erinnerung: $\mathbb{P}(X\in[a,b])=\int_a^b\rho(x)\,dx$.
\subsubsection{Bestimmung von $C$}
\begin{align*}
    &\int_{-\infty}^{\infty}e^{-\frac{(x-\mu)^2}{2\sigma^2}}\,dx
    =\int_{-\infty}^{\infty}e^{-\frac{x^2}{2\sigma^2}}\,dx
    \\\textabove{$y=\frac{x}{\sigma}$}{=}\ &\sigma\int_{-\infty}^{\infty}e^{-\frac{y^2}{2}}\,dy
    \\\textabove{$(*)$}{=}\ &\sigma\cdot\sqrt{2\pi}
\end{align*}
$(*)$:
\begin{align*}
    &\left(\int_{-\infty}^{\infty}e^{-\frac{y^2}{2}}\,dy\right)^2=\int\int e^{-\frac{y^2}{2}}\,dy\ e^{-\frac{z^2}{2}}\,dz
    \\=\ &\int\int e^{-\frac{y^2+z^2}{2}}\,dydz
    \\\textabove{Polark.}{=}\ &\int_{0}^{\infty}2r\pi\cdot e^{-\frac{r^2}{2}}\,dr
    \\=\ &2\pi[\ \ubr{-e^{-\frac{r^2}{2}}}{$\prime=re^{\frac{r^2}{2}}$}\ ]_0^\infty
    \\=\ &2\pi(0-(-1))=2\pi\cdot1
\end{align*}
\newpage
\section{Invarianzen}
\subsection{Erwartungswert}
Wir denken an ein Gl\"uckspiel mit Geldeinsatz, welches wir h\"aufig und unbh\"angig wiederholen. Welchen Gewinn/Verlust erwarten wir im Mittel zu erreichen?
\subsubsection{Definition (Erwartungswert diskret)}
Sei $X\colon\Omega\to\mathbb{R}$ eine Zufallsgr\"o\ss{}e. $|X(\Omega)|$ sei endlich. Dann ist
\[\mathbb{E}(X):=\sum_{a\in X(\Omega)}a\cdot\mathbb{P}(X=a)\]
der \trt{Erwartungswert} von $X$.
\subsubsection{Bemerkung}
Die Definition kann auf abz\"ahlbare Mengen direkt erweitert werden. Hier kann es passieren, dass $\mathbb{E}$ nicht existiert. Beipsiel:
\[\mathbb{P}(X=a)=\frac{C}{a^2},\ a\in\mathbb{N}.\]
F\"ur das Beispiel w\"are $\mathbb{E}(X)=+\infty$, f\"ur andere Beispiele schl\"agt auch diese ``uneigentliche'' Definition fehl.
\subsubsection{Satz}
Falls $\Omega$ endlich ist, gilt:
\[\mathbb{E}(X)=\sum_{\omega\in\Omega}X(\omega)\cdot\mathbb{P}(\{\omega\}).\]
\subsubsection{Beispiel}
Man wirft einen W\"urfel
\[X(1)=-1,\ X(2)=-1,\ X(3)=-1,\ X(4)=-1,\ X(5)=0,\ X(6)=3.\]
Also ist
\[\mathbb{E}(x)=-1\cdot\frac{1}{6}+\ldots3\cdot\frac{1}{6}=(-1)\cdot(\frac{4}{6})+0\cdot\frac{1}{6}+3\cdot\frac{1}{6}\]
\subsubsection{Beweis}
Distributivgesetz:
\[\mathbb{E}=-\int_{-\infty}^{0}V_X(a)\,da+\int_0^\infty1-V_X(a)\,da.\]
\subsubsection{Bemerkung}
Letztere Formel l\"asst uns den Begriff Erwartungswert verallgemeiner. Sowohl der positive als auch der negative Teil existieren immer (zumindest uneigentlich). Einziges m\"ogliches Problem: ``$+\infty\ +\ -\infty$''.
\\~\\
Eine andere M\"oglichkeit besteht mithilfe des Lebesgue-Integrals:
\[\mathbb{E}(x)=\int_{-\infty}^{\infty}x\,d\widetilde{\mathbb{P}}.\]
Man integriert bzgl. des Ma\ss{}es $\widetilde{\mathbb{P}}=\mathbb{P}\cdot X^{-1}$.
\subsubsection{Satz}
F\"ur stetig verteilte Zufallsvariablen $X$, d.h. solch, die eine Dichte besitzen, ist
\[\mathbb{E}(X)=\int x\rho(x)\,dx.\]
\subsubsection{Beweis}
Lassen wir weg.
\subsubsection{Satz}
Der Erwartungswert ist linear, d.h. gegeben $X,Y\colon\Omega\to\mathbb{R}$ Zufallsvariablen, $a\in\mathbb{R}$:
\[\mathbb{E}(aX+Y)=a\mathbb{E}(X)+\mathbb{E}(Y),\]
falls $\mathbb{E}(X)$ und $\mathbb{E}(Y)$ existieren.
\subsubsection{Beweis}
\abc{
\item diskret:
\begin{align*}
\mathbb{E}(X+aY)&=\sum_{\mathclap{z\in(X+aY)(\Omega)}}z\mathbb{P}(X+aY=z)\\
\text{(falls $|\Omega|<\infty$)}\ &=\sum_{\omega\in\Omega}(X+aY)(\omega)\cdot\mathbb{P}(\{\omega\})\\
&=\sum_{\omega\in\Omega}(X(\omega)+aY(\omega))(\omega)\cdot\mathbb{P}(\{\omega\})\\
&=\mathbb{E}(X)+a\mathbb{E}(Y)
\end{align*}
\item Im stetigen Fall kann man $\rho$ als Treppenstufenfunktion approximieren und die Linearit\"at folgt mit a).
}
\subsubsection{Bemerkung}
Die Formel $\mathbb{E}(X\cdot Y)=\mathbb{E}(X)\cdot\mathbb{E}(Y)$ gilt im Allgemeinen \tul{nicht}. Beispiel: $X=1$, falls $\omega=$ Zahl, $X=0$ sonst, $Y=X$.
\subsubsection{Definition (fast sicher)}
Ein Ereignis $A$ hei\ss{}t ``\trt{fast sicher}'', falls $\mathbb{P}(A)=1$.
\subsubsection{Satz}
\abc{
\item Falls $X=Y$ fast sicher, so ist $\mathbb{E}(X)=\mathbb{E}(Y)$.
\item Sei $X\equiv\mu\in\mathbb{R}$, so ist $\mathbb{E}(X)=\mu$.
}
\subsubsection{Beweis}
\abc{
\item $\mathbb{E}(X-Y)$ ist fast sicher $=0\Rightarrow\mathbb{E}(X-Y)=0$,
\item klar.
}
