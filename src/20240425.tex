\pickl{25.04.2024}
\subsubsection{Defintion (Borel-$\sigma$-Algebra)}
Sei $\Omega=\mathbb{R}$. Die von den offenen Teilmengen von $\mathbb{R}$ erzeugte $\sigma$-Algebra hei\ss{}t \trt{Borel-$\sigma$-Algebra}. (alternative Definiton sp\"ater)
\subsection{Wahrscheinlichkeitsma\ss{}}
\subsubsection{Definition (Wahrscheinlichkeitsma\ss{})}
Sei $\Omega$ eine Menge, $\mathcal{A}$ $\sigma$-Algebra. Eine Abbildung $\mathbb{P}\colon\mathcal{A}\to\mathbb{R}$ hei\ss{}t \trt{Wahrscheinlichkeitsma\ss{}} $:\Leftrightarrow$
\abc{
\item $\mathbb{P}(\Omega)=1$,
\item $\mathbb{P}(A)\geq0,\ \forall A\in\mathcal{A}$,
\item $\mathbb{P}(\bigcup_{n=1}^\infty A_n)=\sum_{n=1}^\infty\mathbb{P}(A_n)$, falls $A_n\in\mathcal{A},\ A_n\cap A_m=\emptyset\ \forall n\neq m$.
}
\subsubsection{Bemerkung}
Satz vom Gegenereignis, $\mathbb{P}(\emptyset)=0$ und $\mathbb{P}(A\cup B)=\mathbb{P}(A)+\mathbb{P}(B)-\mathbb{P}(A\cap B)$ gilt weiterhin.
\subsubsection{Definition und Satz (Bedingte Wahrscheinlichkeit)}
Sei $(\Omega,\mathcal{A},\mathbb{P})$ ein Wahrscheinlichkeitsraum, d.h. $\Omega$ Menge, $\mathcal{A}$ zugeh\"orige $\sigma$-Algebra, $\mathbb{P}\colon\AA\to\mathbb{R}$ Wahrscheinlichkeitsma\ss{}.
\\~\\
Sei $A\in\mathcal{A}$ mit $\mathbb{P}(A)\neq0$. Dann ist das auf $A$ \trt{bedingte Wahrscheinlichkeitsma\ss{}} definiert durch
\[
\mathbb{P}_A(B)=\frac{\mathbb{P}(A\cap B)}{\mathbb{P}(A)}.
\]
\subsubsection{Beweis}
(dass dies ein Wahrscheinlichkeitsma\ss{} ist):
\abc{
\item $\mathbb{P}_A(\Omega)=\frac{\mathbb{P}(A\cap\Omega)}{\mathbb{P}(A)}=\frac{\mathbb{P}(A)}{\mathbb{P}(A)}=1.$
\item Z\"ahler $\geq0$, Nenner $>0$ $\Rightarrow$ Behauptung.
\item 
\begin{align*}
    \mathbb{P}_A(\bigcup_{n=1}^\infty A_n)&=\frac{\mathbb{P}(A\ \cap\ \bigcup_{n=1}^\infty A_n)}{\mathbb{P}(A)}\ (A_n\cap A_m=\emptyset,\ n\neq m)\\
    &=\frac{\mathbb{P}(\bigcup_{n=1}^\infty\obr{(A_n\cap A)}{paarweise disjunkt})}{\mathbb{P}(A)}\\
    &=\frac{\sum_{n=1}^\infty\mathbb{P}(A_n\cap A)}{\mathbb{P}(A)}\\
    &=\sum_{n=1}^\infty\mathbb{P}_A(A_n).
\end{align*}
}
\subsubsection{Definition (Unabh\"angigkeit)}
Zwei Ereignisse $A,B$ hei\ss{}en (stochastisch) \trt{unabh\"angig} $:\Leftrightarrow$
\[
\mathbb{P}(A\cap B)=\mathbb{P}(A)\cdot\mathbb{P}(B).
\]
\subsubsection{Bemerkung}
$A$ unabh\"angig von $B$ $\Rightarrow$ $\mathbb{P}_A(B)=\mathbb{P}(B)$ (falls $\mathbb{P}(A)\neq0$).
\subsubsection{Beispiel}
$\Omega=\{1,\ldots,6\}$.
\abc{
\item $A=\{3,4,5,6\},\ B=\{2,4,6\}$ sind unabh\"angig.
\item $\widetilde{A}=\{4,5,6\}$ und $B$ wie oben sind nicht unabh\"angig.
}
\subsubsection{Definition (Limes von Ereignissen)}
Seien $(A_n)_{n\in\mathbb{N}}\subset\mathcal{A}$, $(B_n)_{n\in\mathbb{N}}\subset\mathcal{A}$.
Wir nehmen an:
\bul{
\item $A_n\subset A_{n+1},\ \forall n\in\mathbb{N}$,
\item $B_n\supset B_{n+1},\ \forall n\in\mathbb{N}$.
}
Dann ist
\begin{align*}
    \lim_{n\to\infty}A_n&\ :=\bigcup_{n=1}^\infty A_n,\\
    \lim_{n\to\infty}B_n&\ :=\bigcap_{n=1}^\infty B_n.
\end{align*}
\subsubsection{Korrolar}
$\lim_{n\to\infty}A_n$ und $\lim_{n\to\infty}B_n$ sind Ereignisse, falls $A_n$, $B_n$ Ereignisse sind.
\subsubsection{Beweis}
Definition der $\sigma$-Algebra, bzw. Satz gleich darunter.
\subsubsection{Definition ($\limsup$, $\liminf$)}
Sei $(A_n)_{n\in\mathbb{N}}\subset\mathcal{A}$ Dann ist
\begin{align*}
\limsup_{n\to\infty}A_n&\ :=\bigcap_{k=1}^\infty\bigcup_{n=k}^\infty A_n,\\
\liminf_{n\to\infty}A_n&\ :=\bigcup_{k=1}^\infty\bigcap_{n=k}^\infty A_n.
\end{align*}
\subsubsection{Korollar}
Auch $\limsup$ und $\liminf$ sind Ereignisse (falls $A_n\in\mathcal{A}\ \forall n\in\mathbb{N}$).
\subsubsection{Satz ($\sigma$-Stetigkeit des Wahrscheinlichkeitsma\ss{}es)}
Sei $(A_n)_{n\in\mathbb{N}},(B_n)_{n\in\mathbb{N}}\subset\mathcal{A}$ eine abfallende bzw. ansteigende Folge von Ereignissen ($A_n\subset A_{n+1},\ B_n\supset B_{n+1},\ \forall n$). Dann ist
\abc{
\item $\lim_{n\to\infty}\mathbb{P}(A_n)=\mathbb{P}(\lim_{n\to\infty}A_n)$,
\item $\lim_{n\to\infty}\mathbb{P}(B_n)=\mathbb{P}(\lim_{n\to\infty}B_n)$.
}
\subsubsection{Beweis}
Definiere $C_1:=A_1$. $C_2:=A_1\setminus A_1$, \ldots, $C_n=A_n\setminus A_{n-1}$. Es gilt:
\abc{
\item $\lim_{n\to\infty}A_n=\bigcup_{n=1}^\infty A_n=\bigcup_{n=1}^\infty C_n$.
\item $C_n\cap C_m=\emptyset,\ \forall n\neq m$,
\item $A_n=\bigcup_{k=1}^{n}C_k$.
}
\abc{
\item
\begin{align*}
    \Rightarrow\mathbb{P}(\lim_{\mathclap{n\to\infty}}A_n)&\ \textabove{a)}{=}\mathbb{P}(\bigcup_{n=1}^\infty C_n)
    \\&\ \textabove{\tbf{K}c)}{=}\sum_{n=1}^\infty\mathbb{P}(C_n)
    \\&\ =\lim_{n\to\infty}\sigma_{k=1}^{n}\mathbb{P}(C_k)
    \\&\ \textabove{\tbf{K}c)}{=}\lim_{n\to\infty}\mathbb{P}(\bigcup_{k=1}^nC_k)
    \\&\ \textabove{c)}{=}\mathbb{P}(A_n).
\end{align*}
\myqed{}
\item
\begin{align*}
    \mathbb{P}(\lim_{\mathclap{n\to\infty}}B_n)&\ =1-\mathbb{P}(\lim_{\mathclap{n\to\infty}}B_n^C)
    \\&\ \textabove{Fall a)}{=}\ 1-\lim_{n\to\infty}\mathbb{P}(B_n^C)
    \\&\ =\lim_{n\to\infty}\mathbb{P}(B_n).
\end{align*}
}
