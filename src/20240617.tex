\pickl{17.06.2024}
\subsubsection{Definition (Korrelationskoeffizient)}
F\"ur zwei Zufallsvariablen $X,Y$ nentnt man
\[
\kappa(X,Y):=\frac{\Cov(X,Y)}{\sqrt{\Var X\cdot\Var Y}}
\]
den \trt{Korrelationskoeffizienten}.
\subsubsection{Bemerkung}
$|\kappa(X,Y)|=|\kappa(aX,bY)|,\ \forall a,b\in\mathbb{R}$. Vorzeichenwechsel, genau dann wenn $a$ oder $b$ (exklusiv) negativ.
\subsubsection{Satz}
$|\kappa(X,Y)|\leq1$.
\subsubsection{Beweis (Cauchy-Schwarz)}
Wir benutzen Cauchy-Schwarz:
\[
|\langle X,Y\rangle|\leq\sqrt{\langle X,X\rangle\langle Y,Y\rangle}.
\]
Wir beweisen, dass diese f\"ur alle positiv semi-definiten, symmetrischen Bilinearformen gilt. Denn die Kovarianz ist eine solche.
\begin{itemize}
\item 1. Fall: $\langle x\neq0$ und $\langle y,y\rangle\neq0$. Betrachte $\langle ax\pm by,ax\pm by\rangle$ mit $a,b$ so, dass $\langle ax,ax\rangle=\langle by,by\rangle=1$.
\begin{align*}
\langle ax+by,ax+by\rangle&=a^2\langle x,x\rangle + b^2\langle y,y\rangle\pm 2ab\langle x,y\rangle\\
&=2(1\pm\langle x,y\rangle ab)
\end{align*}
$\Rightarrow 1\pm\langle x,y\rangle ab\geq0\Rightarrow\pm\langle X,Y\rangle\geq\frac{-1}{ab}\Rightarrow\pm\langle x,y\rangle\leq\frac{1}{ab}$. Da $\langle ax,ax\rangle=1\Rightarrow a^2=\frac{1}{\langle x,x\rangle}\Rightarrow\frac{1}{a}=\sqrt{\langle x,x\rangle},\ \frac{1}{b}=\sqrt{\langle x,x\rangle}\Rightarrow\pm\langle x,y\rangle\leq\sqrt{\langle x,x\rangle\langle y,y\rangle}$.
\item $\langle x,x\rangle\neq0,\ \langle y,y\rangle=0$.
\begin{align*}
\langle ax\pm by,ax\pm by\rangle=a^2x^2\pm 2ab\langle x,y\rangle+b^2\langle y,y\rangle
\end{align*}
($b,a$ ist hier ganz allgemein gew\"ahlt), $\Rightarrow a^2x^2\pm2ab\langle x,y\rangle\leq0$. Z.z. $\langle x,y\rangle=0$. WA: $\langle x,y\rangle=\varepsilon>0$. W\"ahle $b=1$ und $a$ hinreichend klein, sowie das negative Vorzeichen $\Rightarrow\lightning$, z.B. $a=\frac{\varepsilon}{\langle x,x\rangle2}$.
\[
a^2x^2-2ab\langle x,y\rangle=\frac{\varepsilon^2}{4\langle x,x\rangle}-\frac{2\varepsilon^2}{2\langle x,x\rangle}\leq0\ \lightning.
\]
\item 3. Fall: $\langle x,x\rangle=\langle y,y\rangle=0$. Betrachte $0\geq\langle x\pm y,x\pm y\rangle=\langle x,x\rangle+\langle y,y\rangle\pm2\langle x,y\rangle\Rightarrow\langle x,y\rangle=0$.
\end{itemize}
\subsubsection{Beweis (des Satzes)}
F\"ur den Vektorraum der beschr\"ankten Zufallsvariablen nehmen wir die Ungleichung von Cauchy-Schwarz her $\Rightarrow$ Behauptung.
\subsubsection{Bemerkung}
Falls $\Var X,\Var Y<\infty$, so ist auch $|\Cov(X,Y)|<\infty$. Dies l\"asst sich mithilfe eines Widersprucharugments zeigen (grob): WA $|\Cov(X,Y)|=\infty\Rightarrow$ durch Abschneiden von $X,Y$:
\[
\widetilde{X}(\omega)=\begin{cases}
X(\omega)&\text{falls }X(\omega)\leq C,\\
C&\text{sonst,}
\end{cases}
\]
$\widetilde{Y}(\omega)$ analog. Man findet f\"ur jedes noch so gro\ss{}e $M\in\mathbb{R}$ ein $C\in\mathbb{R}$, sodass $|\Cov(\widetilde{X},\widetilde{Y})|\geq M$, aber $|\Cov(\widetilde{X},\widetilde{Y})|\leq\sqrt{\Var\widetilde{X}}\sqrt{\Var\widetilde{Y}}\leq\sqrt{\Var X}\sqrt{\Var Y}$. W\"ahle $M>\sqrt{\Var X\Var Y}\Rightarrow\lightning$.
\subsubsection{Satz}
Sei $\Var X\neq 0$, dann ist
\[
\kappa(X,aX+b)=\ubr{\Vz}{Vorzeichen}(a)\in\{\pm 1\}.
\]
(Es reicht, $Y=aX+b$ fast sicher, damit $|\kappa(X,Y)=1|$.)
\subsubsection{Beweis}
\begin{align*}
\kappa(X,aX+b)&=\frac{\Cov(X,aX+b)}{\sqrt{\Var(X)\Var(aX+b)}}\\
&=\frac{a\Cov(X,X)+\Cov(X,b)}{\sqrt{\Var X\cdot a^2\Var X}}\\
&=\frac{a\Var X+0}{|a|\Var X}=\frac{a}{|a|}.
\end{align*}
($|\Cov(X,b)\ubr{\leq}{C.S.}\sqrt{\Var X\ubr{\Var b}{$=0$}}\Rightarrow\Cov(X,b)=0$.)
\subsubsection{Satz}
Falls $|\kappa(X,Y)|=1\Rightarrow Y=aX+b$ fast sicher mit $a,b\in\mathbb{R}$. Es gilt: $a>0\Leftrightarrow\kappa(X,Y)>0$.
\subsubsection{Beispiel}
\[
\Cov(X,Y)=\Cov(X,Y-aX+aX),
\]
w\"ahle $a$, sodass $\obr{Y-aX}{$Z$}$ unkorreliert zu $X$. Da $\Cov(X,Y-aX)=\Cov(X,Y)-a\obr{\Var X}{$\neq0$}$, ist dies immer m\"oglich!
\[
\kappa(X,Y)=\frac{\Cov(X,Z)+\Cov(X,aX)}{\sqrt{\Var X}\sqrt{\Var Z+aX}}=\frac{a\cdot\Var X}{\sqrt{\Var X(\Var Z+a^2\Var X)}}.
\]
Der gr\"o\ss{}tm\"ogliche Wert wird f\"ur $\Var Z=0$ erreicht, dieses ist genau $1\Rightarrow\Var Z=0$. $Z:=Y-aX\Rightarrow Y=aX+\ubr{Z}{fast sicher konstant}$.
