\pickl{15.04.2024}
\trt{In der Stochastik geht es um die Modellierung von Experimenten, deren Ausgang vom Zufall abh\"angt.}
\section{Endliche Wahrscheinlichkeitsr\"aume}
\subsection{Ergebnismenge, Ereignismenge}
\subsubsection{Definition (Ergebnismenge)}
Die Menge $\Omega$, welche die m\"oglichen Ausg\"ange eines Zufallexperiementes beschreibt, dennen wir \trt{Grundmenge} oder \trt{Ergebnismenge}.
\subsubsection{Definition (Ereignismenge)}
Die Potenzmenge $\mathcal{P}(\Omega)$, d.h. die Menge aller Teilmengen $\Omega$, nennen wir \trt{Ereignismenge}.
\subsubsection{Beispiel}
\num{
\item Wir werfen einen W\"urfel.
\bul{
\item $\Omega=\{1,2,\ldots,6\}$,
\item $\mathcal{P}(\Omega)=\{\emptyset,\Omega,\{1\},\ldots,\{1,1\},\ldots\}$,
}
\item Gl\"ucksrad: $\Omega=[0,2\pi[$ beschreibt die m\"oglichen Winkel eines Gl\"uckradspiels. $\mathcal{P}(\Omega)$ ist klar (keine geeignete Ereignismenge, siehe Kapitel 2).
}
\subsection{Wahrscheinlichkeitsma\ss{}}
Das Wahrscheinlichkeitsma\ss{} wird auf der Ereignismenge definiert. Grund: F\"ur \"uberabz\"ahlbare Mengen (Gl\"ucksrad z.B.) haben einzelne Ausg\"ange h\"aufig Wahrscheinlichkeit $0$, obwohl global gesehen existiert ein sinnvolles Wahrscheinlichkeitsma\ss{} (siehe Kapitel 2).
\\~\\
Die Wahrscheinlichkeit quantifiziert die Plausibilit\"at der entsprechenden Ereignisse. Sie gibt die relative H\"aufigkeit an, wie oft ein bestimmtes Ereignis nach sehr h\"aufigen Wiederholen unter identischen Umst\"anden eintritt.
\subsubsection{Definition (Wahrscheinlichkeitsma\ss{})}
Eine Abbildung $\mathbb{P}\colon\mathcal{P}(\Omega)\to\mathbb{R}$ nennt man \trt{Wahrscheinlichkeitsma\ss{}} $:\Leftrightarrow$
\abc{
\item[\tbf{K}a)] $\mathbb{P}(A)\geq0,\ \forall A\subset\Omega$,
\item[\tbf{K}b)]  $\mathbb{P}(\Omega)=1$,
\item[\tbf{K}c)]  $\mathbb{P}(A\cup B)=\mathbb{P}(A)+\mathbb{P}(B),\ \forall A,B\subset\Omega,\ A\cap B=\emptyset$.
}
\subsubsection{Bemerkung}
Die Axiome a)-c) nennt man \trt{Axiome von Kolmogorov} (werden in 2 ebenfalls leicht angepasst).
\subsubsection{Satz}
F\"ur jedes Wahrscheinlichkeitsma\ss{} $\mathbb{P}\colon\mathcal{P}(\Omega)\to\mathbb{R}$ f\"ur beliebiges $\Omega$ gelten:
\abc{
\item $\mathbb{P}(A^C)=1-\mathbb{P}(A),\ \forall A\subset\Omega$.
\item $\mathbb{P}(A)\leq1$.
\item $\mathbb{P}(A\cup B)=\mathbb{P}(A)+\mathbb{P}(B)-\mathbb{P}(A\cap B)$.
}
\subsubsection{Beweis}
\weg
\subsubsection{Bemerkung}
\"Uber die Wahrscheinlichkeiten der Elementarereignisse (d.h. der einelementigen Ereignisse), wird das Wahrscheinlichkeitsma\ss{} eindeutig festgelegt.
\\~\\
Betrachte $\mathbb{P}(A)$ f\"ur $A=\{w_1,w_2,\ldots,w_k\}\subset\Omega$. Durch mehrmaliges Anwenden von \tbf{K}c) erh\"alt man
\[
\mathbb{P}(A)=\sum_{i=1}^{k}\mathbb{P}(\{w_k\}).
\]
\subsubsection{Definition (Wahrscheinlichkeitsraum)}
Das Paar $(\Omega,\mathbb{P})$ nennt man auch \trt{Wahrscheinlichkeitsraum} (auch $(\Omega,\mathcal{P}(\Omega),\mathbb{P})$).

