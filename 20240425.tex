\pickl{25.04.2024}
\subsubsection{Defintiion (Borel-$\sigma$-Algebra)}
Sei $\Omega=\RR$. Die von den offenen Teilmengen von $\RR$ erzeugte $\sigma$-Algebra hei\ss{}t \trt{Borel-$\sigma$-Algebra}. (alternative Definiton sp\"ater)
\subsection{Das Wahrscheinlichkeitsma\ss{}}
\subsubsection{Definition (Wahrscheinlichkeitsma\ss{})}
Sei $\Omega$ eine Menge, $\Acal$ $\sigma$-Algebra. Eine Abbildung $\PP\colon\Acal\to\RR$ hei\ss{}t \trt{Wahrscheinlichkeitsma\ss{}} $:\LRA$
\abc{
\item $\PP(\Omega)=1$,
\item $\PP(A)\geq0,\ \forall A\in\Acal$,
\item $\PP(\bigcup_{n=1}^\infty A_n)=\sum_{n=1}^\infty\PP(A_n)$, falls $A_n\in\Acal,\ A_n\cap A_m=\emptyset\ \forall n\neq m$.
}
\subsubsection{Bemerkung}
Satz vom Gegenereignis, $\PP(\emptyset)=0$ und $\PP(A\cup B)=\PP(A)+\PP(B)-\PP(A\cap B)$ gilt weiterhin.
\subsubsection{Definition und Satz (Bedingte Wahrscheinlichkeit)}
Sei $(\Omega,\Acal,\PP)$ ein Wahrscheinlichkeitsraum, d.h. $\Omega$ Menge, $\Acal$ zugeh\"orige $\sigma$-Algebra, $\PP\colon\AA\to\RR$ Wahrscheinlichkeitsma\ss{}.
\\~\\
Sei $A\in\Acal$ mit $\PP(A)\neq0$. Dann ist das auf $A$ \trt{bedingte Wahrscheinlichkeitsma\ss{}} definiert durch
\meq{\PP_A(B)=\frac{\PP(A\cap B)}{\PP(A)}.}
\subsubsection{Beweis}
(dass dies ein Wahrscheinlichkeitsma\ss{} ist):
\abc{
\item $\PP_A(\Omega)=\frac{\PP(A\cap\Omega)}{\PP(A)}=\frac{\PP(A)}{\PP(A)}=1.$
\item Z\"ahler $\geq0$, Nenner $>0$ $\RA$ Behauptung.
\item 
\meq{
    \PP_A(\bigcup_{n=1}^\infty A_n)&=\frac{\PP(A\ \cap\ \bigcup_{n=1}^\infty A_n)}{\PP(A)}\ (A_n\cap A_m=\emptyset,\ n\neq m)\\
    &=\frac{\PP(\bigcup_{n=1}^\infty\obr{(A_n\cap A)}{paarweise disjunkt})}{\PP(A)}\\
    &=\frac{\sum_{n=1}^\infty\PP(A_n\cap A)}{\PP(A)}\\
    &=\sum_{n=1}^\infty\PP_A(A_n).
}
}
\subsubsection{Definition (Unabh\"angigkeit)}
Zwei Ereignisse $A,B$ hei\ss{}en (stochastisch) \trt{unab\"angig} $:\LRA$
\meq{\PP(A\cap B)=\PP(A)\cdot\PP(B)}.
\subsubsection{Bemerkung}
$A$ unabh\"angig von $B$ $\RA$ $\PP_A(B)=\PP(B)$ (falls $\PP(A)\neq0$).
\subsubsection{Beispiel}
$\Omega=\{1,\ldots,6\}$.
\abc{
\item $A=\{3,4,5,6\},\ B=\{2,4,6\}$ sind unabh\"angig.
\item $\widetilde{A}=\{4,5,6\}$ und $B$ wie oben sind nicht unab\"angig.
}
\subsubsection{Definition (Limes von Ereignissen)}
Seien $(A_n)_{n\in\NN}\subset\Acal$, $(B_n)_{n\in\NN}\subset\Acal$.
Wir nehmen an:
\bul{
\item $A_n\subset A_{n+1},\ \forall n\in\NN$,
\item $B_n\supset B_{n+1},\ \forall n\in\NN$.
}
Dann ist
\meq{
    \lim_{n\to\infty}A_n&\ :=\bigcup_{n=1}^\infty A_n,\\
    \lim_{n\to\infty}B_n&\ :=\bigcap_{n=1}^\infty B_n.
}
\subsubsection{Korrolar}
$\lim_{n\to\infty}A_n$ und $\lim_{n\to\infty}B_n$ sind Ereignisse, falls $A_n$, $B_n$ Ereignisse sind.
\subsubsection{Beweis}
Definition der $\sigma$-Algebra, bzw. Satz gleich darunter.
\subsubsection{Definition ($\limsup$, $\liminf$)}
Sei $(A_n)_{n\in\NN}\subset\Acal$ Dann ist
\meq{
\limsup_{n\to\infty}A_n&\ :=\bigcap_{k=1}^\infty\bigcup_{n=k}^\infty A_n,\\
\liminf_{n\to\infty}A_n&\ :=\bigcup_{k=1}^\infty\bigcap_{n=k}^\infty A_n.
}
\subsubsection{Korollar}
Auch $\limsup$ und $\liminf$ sind Ereignisse (falls $A_n\in\Acal\ \forall n\in\NN$).
\subsubsection{Satz ($\sigma$-Stetigkeit des Wahrscheinlichkeitsma\ss{}es)}
Sei $(A_n)_{n\in\NN},(B_n)_{n\in\NN}\subset\Acal$ eine abfallende bzw. ansteigende Folge von Ereignissen ($A_n\subset A_{n+1},\ B_n\supset B_{n+1},\ \forall n$). Dann ist
\abc{
\item $\lim_{n\to\infty}\PP(A_n)=\PP(\lim_{n\to\infty}A_n)$,
\item $\lim_{n\to\infty}\PP(B_n)=\PP(\lim_{n\to\infty}B_n)$.
}
\subsubsection{Beweis}
Definiere $C_1:=A_1$. $C_2:=A_1\setminus A_1$, \ldots, $C_n=A_n\setminus A_{n-1}$. Es gilt:
\abc{
\item $\lim_{n\to\infty}A_n=\bigcup_{n=1}^\infty A_n=\bigcup_{n=1}^\infty C_n$.
\item $C_n\cap C_m=\emptyset,\ \forall n\neq m$,
\item $A_n=\bigcup_{k=1}^{n}C_k$.
}
\abc{
\item
\meq{
    \RA\PP(\lim_{\mathclap{n\to\infty}}A_n)&\ \textabove{a)}{=}\PP(\bigcup_{n=1}^\infty C_n)
    \\&\ \textabove{\tbf{K}c)}{=}\sum_{n=1}^\infty\PP(C_n)
    \\&\ =\lim_{n\to\infty}\sigma_{k=1}^{n}\PP(C_k)
    \\&\ \textabove{\tbf{K}c)}{=}\lim_{n\to\infty}\PP(\bigcup_{k=1}^nC_k)
    \\&\ \textabove{c)}{=}\PP(A_n).
}
\myqed{}
\item
\meq{
    \PP(\lim_{\mathclap{n\to\infty}}B_n)&\ =1-\PP(\lim_{\mathclap{n\to\infty}}B_n^C)
    \\&\ \textabove{Fall a)}{=}\ 1-\lim_{n\to\infty}\PP(B_n^C)
    \\&\ =\lim_{n\to\infty}\PP(B_n).
}
}
